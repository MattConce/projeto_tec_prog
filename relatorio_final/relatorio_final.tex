\documentclass[12pt,a4paper]{article}

\usepackage[utf8]{inputenc}
\usepackage[T1]{fontenc}
\usepackage{amsmath}
\usepackage[brazilian]{babel}
\usepackage{multicol} 
\usepackage{amsfonts}
\usepackage{hyperref}
\usepackage{graphicx}
\usepackage{caption}
\usepackage{subcaption}
\usepackage{float} % necessário para manter as imagens no lugar certo
%\usepackage{wrapfig}
\title{{\bf Universidade de São Paulo} \\MAC 0218 - Técnicas de Programação 2 \\ Relatório da Fase Final do Projeto}
\author{Leonardo Ikeda     nº USP: 10262822 \\
	    Marcelo Schmitt    nº USP: 9297641\\
	    Matheus Conceição  nº USP: 10297672}
\begin{document}
\date{}
\maketitle

%Deve ser entregue no paca um relatório sucinto do que foi feito, quais os próximos passos  e das principais dificuldades encontradas. 

\noindent Funcionalidades implementadas até agora:
\begin{itemize}
	\item Tela de boas vindas (acessível pela url \url{http://localhost:3000/})
	\item Cadastro de usuário (\url{http://localhost:3000/signup})
	\item Visualização básica de cada usuário cadastrados (\url{http://localhost:3000/users/1} para informações do usuário de id 1 por exemplo)
	\item Listagem de usuário na tela \url{http://localhost:3000/users}
	\item O sistema de login agora permite que apenas usuários logados façam depoimentos. A criação de um email e senha para acesso ao sistema pode ser feita na página \url{http://localhost:3000/signup}
	\item Registro de depoimentos dos usuário. Quando logado um usuário pode deixar um novo depoimento acessando a tela \url{http://localhost:3000/} e clicando no botão Postar logo abaixo da caixa de texto para digitação do depoimento.
	\item Os depoimentos dos usuários podem ser vistos na página de perfil de cada usuário (para o primeiro usuário cadastrado essa página é \url{http://localhost:3000/users/1})
	\item Cadastro do tipo sanguíneo e tipo de câncer que um usuário possa ter (pode ser feito na página \url{http://localhost:3000/register})
	\item Os usuários podem complementar seu cadastro criando um registro onde ficam salvos o seu tipo sanguíneo, tipo de câncer, se tem interesse em ser doador de sangue, se tem interesse em ser doador de medula óssea, e em que local deseja fazer possíveis doações \url{http://localhost:3000/register}.
	\item O envio de e-mail informativos para usuários que tenham feito um registro complementando o seu cadastrado traz informações específicas sobre o tipo de câncer cadastrado no registro do usuário.
	\item A interface do sistema foi melhorada para seguir o mesmo sistema de cores da página combateaocancer \url{http://www.combateaocancer.com/}.
	\item A tela de informações gerais sobre câncer foi deixada na tela principal de ajuda \url{http://localhost:3000/help}

\end{itemize}

\noindent Dificuldades encontradas:
\begin{itemize}
	\item Configuração e uso do RSpec com FactoryBot
	\item Elaboração dos testes automatizados
	\item Configuração das rotas para cada uma das páginas
	\item Pretendiamos permitir que os usuários incluissem uma imagem de perfil mas não conseguimos implementar totalmente essa funcionalidade
	\item Outra funcionalidade que não conseguimos implementar foi a funcionalidade de publicar os depoimentos como posts no facebook
	\item Formulação de testes para o envio de emails
\end{itemize}

\end{document}
