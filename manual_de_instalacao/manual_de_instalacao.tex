\documentclass[12pt,a4paper]{article}

\usepackage[utf8]{inputenc}
\usepackage[T1]{fontenc}
\usepackage{amsmath}
\usepackage[brazilian]{babel}
\usepackage{multicol} 
\usepackage{amsfonts}
\usepackage{hyperref}
\usepackage{graphicx}
\usepackage{caption}
\usepackage{subcaption}
\usepackage{float} % necessário para manter as imagens no lugar certo
%\usepackage{wrapfig}
\title{{\bf Universidade de São Paulo} \\MAC 0218 - Técnicas de Programação 2 \\ Manual de Instalação e Uso da Segunda Fase do Projeto}
\author{Leonardo Ikeda     nº USP: 10262822 \\
	    Marcelo Schmitt    nº USP: 9297641\\
	    Matheus Conceição  nº USP: 10297672}
\begin{document}
\date{}
\maketitle

Instruções:

Considerando que foram feitos os passos de instalação contidos nas primeiras três páginas da apostila de rails do projesor Marco Dimas Gubitoso.

\begin{enumerate}
	\item Instale SGBD (Sistema Gerenciador de Banco de Dados) postgresql seguindo as instruções em (\url{https://www.postgresql.org/download/}).
	\item Baixe os arquivos do repositóri com executando o comando\\ \texttt{git clone https://github.com/MattConce/projeto\_tec\_prog.git}\\ ou com o comando\\ \texttt{git clone git@github.com:MattConce/projeto\_tec\_prog.git}
	\item Acesse o diretório onde o projeto foi clonado pelo terminal e execute o comando \texttt{bundle install}
	\begin{enumerate}
		\item Se \texttt{bundle install} falhar tente instalar a libpg-dev com \\ \texttt{sudo apt-get install libpq-dev}
		\item Execute o comando \\
		\texttt{gem install pg} \\
		no diretório do projeto
		\item Execute \texttt{bundle install} novamente
	\end{enumerate}
	\item Crie e configure o usuário de acesso ao banco de dados
	\begin{enumerate}
		\item Conecte no banco com esse usuário\\
		\texttt{sudo su - postgres}
		\item Acesse o SGBD com\\
		\texttt{psql}
		\item Altere a senha do administrador\\
		\texttt{ALTER USER postgres WITH ENCRYPTED PASSWORD '<sua\_senha>';}
		\item Desconecte do SGBD\\
		\texttt{$\backslash$q}
		\item Volte ao seu usuário\\
		\texttt{exit}
		\item Edite o arquivo pg\_hba . conf\\
		\texttt{sudo vim / etc / postgresql /9.6/ main / pg\_hba . conf}
		\item Na linha referente ao postgres (padrão é linha 80) altere o arquivo o modo de autenticação de
		\begin{verbatim}
		# Database administrative login by Unix domain socket
		local   all             postgres                      peer
		\end{verbatim}
		para
		\begin{verbatim}
		# Database administrative login by Unix domain socket
		local   all             postgres                      md5
		\end{verbatim}
		\item Feche o arquivo e reinicie o SGBD com \\
		\texttt{sudo /etc/init.d/postgresql restart}
		\item Crie o usuário de acesso ao banco de dados com  \\
		\texttt{sudo -u postgres createuser combateaocancer -s}.\\
		Se isso falhar tente:
		\begin{enumerate}
			\item \texttt{sudo su - postgres}
			\item \texttt{psql}
			\item \texttt{CREATE ROLE combateaocancer WITH ENCRYPTED PASSWORD 'senha';}
			\item \texttt{ALTER ROLE combateaocancer WITH LOGIN CREATEDB;}
			\item \texttt{CREATE DATABASE combateaocancer;}
			\item \texttt{alter database combateaocancer owner to combateaocancer;}
			\item Saia do SGBD com \\
			\texttt{$\backslash$q}
			\item Volte ao seu usuário\\
			\texttt{exit}
			\item Pule para o item (n)
		\end{enumerate}
		\item Acesse o SGBD como admin com \\
		\texttt{psql -U postgres}
		\item Altere a senha de acesso ao BD executando \\
		\texttt{alter user combateaocancer with encrypted password 'senha';}
		\item Crie um banco de dados para o usuário \\
		\texttt{CREATE DATABASE combateaocancer;}
		\item Altere as permições com\\
		\texttt{alter database combateaocancer owner to combateaocancer;}
		\item Edite o mondo de conexão com o BD\\
		\texttt{sudo gedit /etc/postgresql/9.6/main/pg\_hba.conf}
		\item na linha referente a conexão local (normalmente a linha 90), abaixo da linha:
		\begin{verbatim}
		# Database administrative login by Unix domain socket
		local   all             postgres                           peer
		\end{verbatim}
		para
		\begin{verbatim}
		# Database administrative login by Unix domain socket
		local   all             postgres                           md5
		\end{verbatim}
		\begin{verbatim}
		# "local" is for Unix domain socket connections only
		\end{verbatim}
		insira:
		\begin{verbatim}
		local   all             combateaocancer                    md5
		\end{verbatim}
		seu arquivo deve ficar assim:
		\begin{verbatim}
		# "local" is for Unix domain socket connections only
		local   all             combateaocancer                    md5
		local   all             all                                peer
		# IPv4 local connections:
		host    all             all             127.0.0.1/32       md5
		# IPv6 local connections:
		host    all             all             ::1/128            md5
		\end{verbatim}
		\item Feche o arquivo e reinicie o BD com \\
		\texttt{sudo /etc/init.d/postgresql restart}
	\end{enumerate}
	\item Execute o comando\\
	\texttt{rake db:create}
	\begin{enumerate}
		\item Se \texttt{rake db:create} terminar com erro tente executar os comandos\\
		\texttt{bin/rails db:migrate RAILS\_ENV=development}\\
		e / ou\\
		\texttt{bin/rails db:migrate RAILS\_ENV=test}
	\end{enumerate}
	\item Execute \texttt{bin/rails server}
	\item Acesse \url{http://localhost:3000/}
\end{enumerate}





\end{document}
