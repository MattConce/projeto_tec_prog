\documentclass[12pt,a4paper]{article}

\usepackage[utf8]{inputenc}
\usepackage[T1]{fontenc}
\usepackage{amsmath}
\usepackage[brazilian]{babel}
\usepackage{multicol} 
\usepackage{amsfonts}
\usepackage{hyperref}
\usepackage{graphicx}
\usepackage{caption}
\usepackage{subcaption}
\usepackage{float} % necessário para manter as imagens no lugar certo
%\usepackage{wrapfig}
\title{{\bf Universidade de São Paulo} \\MAC 0218 - Técnicas de Programação 2 \\ Relatório da Segunda Fase do Projeto}
\author{Leonardo Ikeda     nº USP: 10262822 \\
	    Marcelo Schmitt    nº USP: 9297641\\
	    Matheus Conceição  nº USP: 10297672}
\begin{document}
\date{}
\maketitle

%Deve ser entregue no paca um relatório sucinto do que foi feito, quais os próximos passos  e das principais dificuldades encontradas. 

\noindent Funcionalidades implementadas até agora:
\begin{itemize}
	\item Tela de boas vindas (acessível pela url \url{http://localhost:3000/})
	\item Cadastro de usuário (\url{http://localhost:3000/signup})
	\item Visualização básica de cada usuário cadastrados (\url{http://localhost:3000/users/1} para informações do usuário de id 1 por exemplo)
\end{itemize}

\noindent Funcionalidades previstas:
\begin{itemize}
	\item Melhoria da visualização de cada usuário.
	\item Cadastro de depoimentos
	\item Tela de login
	\item Tela de informações gerais sobre câncer
	\item Envio de e-mail informativos para usuários cadastrados
	\item Cadastro para doaçao de medula e sangue
\end{itemize}

\noindent Dificuldades encontradas:
\begin{itemize}
	\item Configuração do postges: foi bem chato configurar o usuário do postgres para permitir a aplicação ter acesso ao banco de dados. Isso se refletiu no manual de instalação.
	\item Estabelecer as relaçoes entre os modelos do rails, tanto que acebei deixamos apenas o modelo do usuario pronto nessa fase.
\end{itemize}

\end{document}
